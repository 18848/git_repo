\chapter{Introdução}
\noindent<Ao iniciar um novo capítulo, deve ser inserida uma quebra de secção “página ímpar”.>\\
Um capítulo pode incluir 2 ou mais subcapítulos.>\\
\noindent\textit{A introdução é a “porta de entrada” da dissertação, por isso, deverá ser redigida com redobrado cuidado. Este capítulo, o primeiro, deverá apresentar e enquadrar o tema de forma genérica, reportando por exemplo dados estatísticos genéricos. Deve descrever-se com pormenor a questão que se vai analisar e referir a importância dessa análise...\\
A parte final da introdução deve incluir uma descrição do conteúdo de todas as secções da tese, referindo-se a estrutura adotada ao longo da mesma. A título de exemplo, pode incluir-se parágrafos iniciados por “No capítulo 5 adota-se uma perspetiva de..... Finalmente, o último capítulo da tese diz respeito às conclusões da dissertação, onde se...”\cite{Inform}.\\
}
\textit{<Não esqueça de incluir a bibliografia no último ponto, a qual deverá seguir o formato APA.}\\
\textit{O título da bibliografia e dos anexos é formatado de forma semelhante aos capítulos de texto, mas sem numeração. Deve ser inserida uma quebra de secção “página ímpar” antes do título, tal como se apresenta no modelo.>}\\
\textit{O ficheiro de bibliografia do mendeley deverá de ser colocado na diretoria raíz com o nome Proposal.bib}
